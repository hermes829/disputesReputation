\section{Reviewer 1}

\subsection{Major Comments}

\begin{enumerate}
	\item  I remain unconvinced that the main methodological problem -- the use of an, in my view, inadequate dataset as dependent variable -- can be remedied at this stage. A totally different study design would be required, using meaningful data that reflects companies' confidence as claimed. You are not convinced by the literature that explains why the dataset is inadequate, and I don't find the your response satisfying. If Kerner's example of FDI data for Moldova and Poland does not convince you that FDI data is not a valid measure of confidence to put money in a potential perilous place, I wonder what does. I recognise that there are dozens of published studies that use this dataset to assess something in the context of BITs, but that does not make it any better. It only shows that research teams often remain homogeneous and do not cross disciplines (in this case involving someone who has a solid understanding of investment treaties and associated institutions, or on the way FDI data are compiled). I am also not convinced by the responses to the other aspects. The imitation of shortcomings of earlier studies makes sense if the sole purpose is to show that these other authors were right or wrong (on other grounds) -- as you do for Allee \& Peinhardt --, but does not produce genuinely useful findings for the research hypothesis. If the purpose is to show that others were wrong, that should be flagged. If the purpose is to make a valid statement on an observable fact, then imitation of earlier errors is a rather poor approach. There is dire need for the study of the effects of IIAs and associated institutions, but research should move away from the oft-repeated BITs-and-FDI-data setup that almost totally dominates the field and is flawed on data for both variables (the Energy Charter Treaty has clocked up almost 15\% of cases now, and NAFTA follows closely, so keeping to ignore these treaties and cases is almost as unforgivable as using FDI data). That being said, an interesting research question that merits a close look (using, for instance, firm-level data). I also truly believe that the manuscript has greatly improved.
	\begin{itemize}
		\item \textcolor{blue}{ \emph{
		We thank the reviewer for her/his comments. We do not necessarily disagree with the reviewer that some of the ways in which FDI have been employed broadly in the IPE literature and specifically in the IIAs/ISDS are problematic. However, we would point to the justifications that we listed earlier when explaining why we used investment flow data. More importantly, the main focus of our paper involves a more direct test of the effect that involvement in ISDS procedures has on reputation via the ICRG, Heritage, and Fraser indices. This is also why we have already noted in the paper that the change in a country's reputation as welcoming for foreign investment may not be adequately reflected in investment flows. However, to make sure that the reviewer's concerns with respect to FDI are noted in the manuscript we have added a discussion of this issue in a footnote at the end of our analysis involving FDI. 
		}}
	\end{itemize}
\end{enumerate}
