\PassOptionsToPackage{table}{xcolor}
\documentclass[12pt,onesided]{amsart} 
\usepackage{natbib}
\usepackage{amsmath}
\usepackage{pifont}
\usepackage[multiple]{footmisc}
\usepackage{subfigure}
\usepackage[T1]{fontenc}
\usepackage{xcolor}
\usepackage{multirow,multicol}
\usepackage{caption}
\usepackage{booktabs}
\usepackage{color}
\usepackage{setspace}
\usepackage{colortbl}
\usepackage{changebar}
\usepackage{dcolumn}
\usepackage{graphicx}
\usepackage{tikz}
\usepackage[section]{placeins}
\setkeys{Gin}{width=\linewidth,totalheight=\textheight,keepaspectratio}
\graphicspath{{graphics/}{../graphics/}}

\usepackage{hyperref}

\usepackage{geometry}
\geometry{verbose,tmargin=1in,bmargin=1in,lmargin=1in,rmargin=1in}
\usepackage{booktabs}
\usepackage{units}
\usepackage{fancyvrb}
\fvset{fontsize=\normalsize}
\usepackage{lipsum}

\usepackage{footnote}
\usepackage{url} 
% \usepackage[nolists,tablesfirst]{endfloat} % places floats at end
\usepackage{nameref}

\begin{document}

%\maketitle
\thispagestyle{empty}

\begin{center}
{\sc \large The Reputational Impact of Investor-State Disputes}
\end{center}

% \vspace{10mm}

% \begin{center}
% {\sc Shahryar Minhas \& Karen Remmer}\footnote{Shahryar Minhas and Karen Remmer, Department of Political Science, Duke University, Durham, NC 27708 (shahryar.minhas@duke.edu, karen.remmer@duke.edu).}
% \end{center}

\vspace{20mm}

\begin{quote}
\noindent \textbf{Abstract}. To what extent do alleged violations of international commitments damage state reputation? This paper explores this question with specific reference to investor-state disputes arising under the protection of international investment agreements. Existing theory assumes that institutionalization of international commitments raises the ex post costs of defection, including reputational damage, thereby creating strong incentives for state compliance. We modify this expectation by arguing that the consequences of claimed treaty violations are contingent on institutional rules and information. Drawing on original analyses of the impact of investor-state disputes on  FDI flows as well as  investment reputation, we show that the impact of investor-state disputes has been relatively marginal until quite recently, with reputational effects varying with dispute visibility and the relative transparency of dispute settlement processes. We validate our argument that the reputational effects of disputes have only manifested in recent years using three alternative indicators. The central implication of these findings for the broader body of literature on international institutions is that reputational mechanisms for effective treaty enforcement cannot be taken as given but instead need to be explored on the basis of a more nuanced approach addressing the pivotal issues of institutional design and related information costs.
\end{quote}
 
\vspace{10mm}

\begin{center}
Word Count: 8,541
\end{center}

\end{document}