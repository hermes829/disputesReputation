\section{Reviewer 1}

\subsection{Major Comments}

\begin{enumerate}
	\item To what extent is the basic assumption plausible that ISDS claims tarnish reputation? A series of studies have shown that awareness of IIAs and more so of claims is very limited even among foreign investors - why would claims impact on reputation? Among well-informed investors, it has become known that a good number of claims are entirely without merits, so the fact that a state is hit by a claim does not necessarily mean that the state has done anything reproachable (the institution that has most suffered in terms of reputation lately is ISDS itself). There are so many more interactions between businesses and governments (and so few claims, overall and against individual countries) that contribute to reputation, that noise probably covers any sign that could come out of investment treaty claims. In all, the findings are likely to be artifacts. 
	\begin{itemize}
		\item \textcolor{blue}{ \emph{
		The reviewer asks ``To what extent is the basic assumption plausible that ISDS claims tarnish reputation?'' We agree with the reviewer and devote pp. 3 to 7 at the front end of the paper to discussing and challenging the plausibility of prior research findings claiming that investment treaty arbitration has a reputational impact. The assumption re ISDS claims is plausible BECAUSE of these prior studies. IF they are wrong (and we show they are), then the assumption becomes much less plausible.
		}}
	\end{itemize}
	\item One key problem of the design is the use of FDI data (and in particular the highly volatile flow data) for the econometric analysis. Although often repeated, the use of this dataset is highly problematic for the purpose of the study, as has been recognized for a decade now (see first Robert E. LIPSEY (2007), ``Defining and measuring the location of FDI output'' Sjoerd Beugelsdijk/Jean-Francois Hennart/Arjen Slangen/Roger Smeets (2010), ``Why and how FDI stocks are a biased measure of MNE affiliate activity''; and later Andrew Kerner (2014), ``What We Talk About When We Talk About Foreign Direct Investment'' and Andrew Kerner/Jane Lawrence (2012)). Many econometric studies that seek to assess treaty effects still use this data for lack of other available data or due to ignorance, but this is not a good reason to use this data without any discussion on its validity and implications for the exercise. 
	\begin{itemize}
		\item \textcolor{blue}{ \emph{ 
		We do not agree that FDI flow data are inappropriate. The literature cited to make this point is simply not relevant. a) Kerner and Lawrence criticize the use of flow data on the grounds that foreign capital investment takes different forms and the level of political risk varies accordingly. We're interested in foreign investor confidence, however, not  vulnerability to political risk. Fixed capital may be more vulnerable than more volatile forms of investment to political risk, but that is irrelevant for the purposes of our analysis. All kinds of investment flows, whether they end up in the stock market or investment in land, are indicators of business confidence. b) Kerner argues that FDI flow data ``measure the impact that MNCs have in the host country's capital account.'' That is precisely what we and others are interested in-NOT MNC assets, value added, sales or employment. The reason is that we are not interested in the relative importance or the impact of foreign investments on the domestic economy. We agree with Kerner that ``Different research questions demand different conceptualizations.'' FDI sales, assets, employment, etc. do not address questions about the impact of disputes on investment reputation. c) Beugelsdijk, Hennart, Slangen and Smeets challenge the appropriateness of using FDI stock data to measure the value-adding activity of MNCs in host countries, We are NOT using FDI stock data and we are NOT interested in measuring the value-adding activity of MNCs in host countries. d) Lipsey (2007) argues that measures of flows and stocks don't measure the relative importance of FDI in terms of employment, output, etc. to an economy or the distribution of FDI by industry. We are not interested in either. The key point is that there is nothing intrinsically problematic with using investment flow data-it just depends on what one is trying to explain.		
		}}
	\end{itemize}
	\item Also: Why is the number of claims not normalised against the volume of investment that individual countries receive overall (a country that receives little investment from anywhere is unlikely to be exposed to a lot of claims, while a country that attracts a lot of foreign investment would normally be more likely to get hit by claims); those that perceive the reputation can be assumed to factor this in. Also, not all countries have concluded IIAs with countries from which they receive meaningful amounts of investment, so the exposure to such claims is very different. This is also a fact that would be known to those that know about claims. 
	\begin{itemize}
		\item \textcolor{blue}{ \emph{
		Three reasons: a) In the interest of making scholarship cumulative, we are using the same indicators as used by those whose claims we are challenging. b) The commentator has a hypothesis that more investment means more claims, but this is merely a hypothesis and almost certainly wrong. Recipients of unusually low levels of  FDI (e.g., Moldova, and Turkmenistan) have had more disputes lodged against them (20) than four of the top five recipients of FDI (UK, Hong Kong, China, and Germany-a total of 7 disputes). c) We are interested in addressing the claims of prior research. Additionally, the point about having concluded or not concluded IIAs from FDI source countries is not very pertinent. "Treaty shopping" is an intrinsic feature of investment treaty arbitration. Philip Morris, for example, brought a legal claim versus Australia under an Australian-Hong Kong BIT, a legal claim against Uruguay under a Swiss-Uruguayan treaty, etc..
		}}
	\end{itemize}
	\item At FN46, the authors express the surprising view that ``we expect the number of ratified BITs to be positively related to reputation''. Many hold that the opposite is likely to be the case (at least among developing economies, but the text is unclear on whether advanced economies are included in the statement, given the statement after FN53). BITs would more likely be used by states to compensate for mixed reputation to international investors - see, e.g., the papers cited earlier at FN11 and FN12. 
	\begin{itemize}
		\item \textcolor{blue}{ \emph{ 
		Our focus here is not to understand the role that BITs may have on reputation. We are aware that there is much debate in the literature about the positive or null role that BITs may have on FDI and reputation. We are not focused on engaging with this debate in our paper. 
		}}
	\end{itemize}
\end{enumerate}

\subsection*{Other Comments}

\begin{enumerate}
	\item When the distinctive features of ICSID are described, the second item does not set ICSID apart. In fact, all arbitration institutions and rules, in combination with the IIAs, provide binding and enforceable awards. The legal authority of ICSID, if such a thing exists, appears irrelevant, as decisions are taken by the same kind of arbitrators that also adjudicate disputes under other rules and institutions. ICSID itself only facilitates the adjudication process. 
	\begin{itemize}
		\item \textcolor{blue}{ \emph{
		We have changed the length and language of our discussion of the ICSID and added a citation buttressing our points about its distinctiveness.
		}}
	\end{itemize}
	% \clearpage
	\item Slicing off upper income nations (at FN40) for unspecific reasons (``significantly different role in the system'' - why?) is not a plausible and satisfying way to address this issue. The fact that advanced economies are increasingly defendants of treaty claims (without their reputation being tarnished) is interesting and questions the basic assumptions. Canada is a case in point: It got hit by a large number of claims, of which it lost some, and still does not have a ``bad'' reputation for foreign investors. How would you explain this fact?	
	\begin{itemize}
		\item \textcolor{blue}{ \emph{ 
		We slice off upper income nations to make our research comparable to those of others. This is the same as Allee and Peinhart, Aisbett, Busse and Nunnenkamp, etc. have done. We cannot address the robustness of their findings if we use an entirely different case base. 
		}}
		\item \textcolor{blue}{ \emph{ 
		However, to highlight the robustness of our results we reestimate Table 4 (our models of investment profile) below with upper income countries included. As you can see the results remain similar. 
		}}

			\begin{table}[ht]
			\centering
			\begingroup\footnotesize
			\begin{tabular}{lr@{} lr@{}lr@{}lr@{} lr@{}lr@{}lr@{}}
			Variable && Model 1 && Model 2 && Model 3 && Model 4 && Model 5 && Model 6 \\ 
			  \hline
			\hline
			ICSID (past 2 years) & $-0$&$.184^{\ast\ast}$ &&  &&  &&  &&  &&  \\ 
			   & (0&.058) &&  &&  &&  &&  &&  \\ 
			  Not ICSID (past 2 years) &  && 0&.03 &&  &&  &&  &&  \\ 
			   &  && (0&.113) &&  &&  &&  &&  \\ 
			  ICSID (past 5 years) &  &&  && $-0$&$.126^{\ast\ast}$ &&  &&  &&  \\ 
			   &  &&  && (0&.046) &&  &&  &&  \\ 
			  Not ICSID (past 5 years) &  &&  &&  && 0&.024 &&  &&  \\ 
			   &  &&  &&  && (0&.084) &&  &&  \\    
			  Cumulative ICSID$_{t-1}$ &  &&  &&  &&  && $-0$&$.092^{\ast\ast}$ &&  \\ 
			   &  &&  &&  &&  && (0&.034) &&  \\ 
			  Cumulative Not ICSID$_{t-1}$ &  &&  &&  &&  &&  && -0&.006 \\ 
			   &  &&  &&  &&  &&  && (0&.055) \\ 
			  \%$\Delta$ GDP (past  years) & $0$&$.024^{\ast\ast}$ & $0$&$.023^{\ast\ast}$ & $0$&$.025^{\ast\ast}$ & $0$&$.023^{\ast\ast}$ & $0$&$.025^{\ast\ast}$ & $0$&$.023^{\ast\ast}$ \\ 
			   & (0&.008) & (0&.008) & (0&.008) & (0&.008) & (0&.008) & (0&.008) \\ 
			  Ln(GDP per capita) (past  years) & $0$&$.778^{\ast}$ & $0$&$.8^{\ast}$ & $0$&$.784^{\ast}$ & $0$&$.791^{\ast}$ & $0$&$.847^{\ast}$ & $0$&$.811^{\ast}$ \\ 
			   & (0&.39) & (0&.394) & (0&.391) & (0&.397) & (0&.399) & (0&.402) \\ 
			  Ln(Pop.) (past  years) & $2$&$.602^{\ast\ast}$ & $2$&$.535^{\ast\ast}$ & $2$&$.637^{\ast\ast}$ & $2$&$.534^{\ast\ast}$ & $2$&$.703^{\ast\ast}$ & $2$&$.541^{\ast\ast}$ \\ 
			   & (0&.381) & (0&.387) & (0&.38) & (0&.386) & (0&.381) & (0&.387) \\ 
			  Ln(Inflation) (past  years) & $-0$&$.224^{\ast\ast}$ & $-0$&$.209^{\ast}$ & $-0$&$.234^{\ast\ast}$ & $-0$&$.208^{\ast}$ & $-0$&$.252^{\ast\ast}$ & $-0$&$.209^{\ast}$ \\ 
			   & (0&.082) & (0&.086) & (0&.08) & (0&.086) & (0&.079) & (0&.085) \\ 
			  Internal Stability (past  years) & $0$&$.169^{\ast\ast}$ & $0$&$.171^{\ast\ast}$ & $0$&$.169^{\ast\ast}$ & $0$&$.171^{\ast\ast}$ & $0$&$.167^{\ast\ast}$ & $0$&$.17^{\ast\ast}$ \\ 
			   & (0&.037) & (0&.037) & (0&.036) & (0&.037) & (0&.036) & (0&.037) \\ 
			  External Stability (past  years) & -0&.052 & -0&.049 & -0&.055 & -0&.048 & -0&.058 & -0&.049 \\ 
			   & (0&.038) & (0&.038) & (0&.038) & (0&.038) & (0&.038) & (0&.038) \\ 
			  Ratif. BITs (past  years) & $0$&$.041^{\ast\ast}$ & $0$&$.038^{\ast\ast}$ & $0$&$.042^{\ast\ast}$ & $0$&$.038^{\ast\ast}$ & $0$&$.042^{\ast\ast}$ & $0$&$.039^{\ast\ast}$ \\ 
			   & (0&.006) & (0&.006) & (0&.006) & (0&.006) & (0&.006) & (0&.006) \\ 
			  Capital Openness (past  years) & $0$&$.207^{\ast\ast}$ & $0$&$.211^{\ast\ast}$ & $0$&$.204^{\ast\ast}$ & $0$&$.211^{\ast\ast}$ & $0$&$.189^{\ast\ast}$ & $0$&$.211^{\ast\ast}$ \\ 
			   & (0&.069) & (0&.071) & (0&.068) & (0&.071) & (0&.07) & (0&.071) \\ 
			  Polity (past  years) & $0$&$.011^{\ast\ast}$ & $0$&$.011^{\ast\ast}$ & $0$&$.011^{\ast\ast}$ & $0$&$.011^{\ast\ast}$ & $0$&$.011^{\ast\ast}$ & $0$&$.011^{\ast\ast}$ \\ 
			   & (0&.003) & (0&.003) & (0&.003) & (0&.003) & (0&.003) & (0&.003) \\ 
			   \hline
			n && 3296 && 3296 && 3295 && 3295 && 3296 && 3296 \\ 
			  N && 126 && 126 && 126 && 126 && 126 && 126 \\ 
			   \hline
			\hline
			\end{tabular}
			\endgroup
			\caption{Regression on investment profile using country fixed effects, robust standard errors in parentheses. $^{**}$ and $^{*}$ indicate significance at $p< 0.05 $ and $p< 0.10 $, respectively.} 
			\end{table}
			\FloatBarrier

	\end{itemize}
	\item Figure 1 arguably has a normalisation problem: the overall number of newspaper articles referenced on LexisNexis probably goes up every year, so a potentially useful measure of public attention would be the priority - relative frequency of mentioning - that ``ICSID'' gets in the news. This could be measured as a percentage of articles mentioning ``ICSID'' in a given year in all referenced articles in that given year. It is likely to show quite a different graph, and is more meaningful that the absolute number. The axis-title ``frequency'' should also be amended, as a frequency cannot be expressed in absolute numbers (``occurrences'' would probably be better for this graph, but ``proportion'' or ``share of'' would be probably required if the graph was normalised as proposed here. 	
	\begin{itemize}
		\item \textcolor{blue}{ \emph{ 
		We have changed the axis-title to ``Occurrences''. However, we would note that if the increasing mentions of ICSID is simply an artifact of more articles every year, then we would note that we would expect a somewhat linear relationship from around 2000 onwards. Instead we see very few mentions of ICSID at all up until 2006, and then a substantial rise after 2010. This doesn't correspond with a story of the mentions just being tied to increasing amounts of media on LexisNexis.
		}}
	\end{itemize}
	\item Why is population size (FN48) assumed to be positively correlated with international reputation? The countries with the largest populations do not normally feature on top in this regard. In the top 20, only three countries would be considered by most as having an above average reputation in foreign investors' eyes, and these are all advanced economies (and excluded from the study). 
	\begin{itemize}
		\item \textcolor{blue}{ \emph{
		We are addressing claims based on prior research.
		}}
	\end{itemize}
	\item A large number of claims are now brought under multilateral arrangements (e.g. Energy Charter Treaty, NAFTA, CAFTA-DR) that have almost identical features as BITs and FTAs with respect to investment protection. Focusing exclusively on bilateral arrangements is an interesting choice in this regard (but admittedly, UNCTAD does not provide the required dataset off the shelf). 		
	\begin{itemize}
		\item \textcolor{blue}{
		\emph{ 
		The vast majority of claims brought to ICSID are based on BITs. We would note that obtaining data on multilateral agreements is not very difficult either. The UNCTAD Investment Policy Hub website has information on multilateral agreements that include IIAs. Additionally, the DESTA database can also be used to ascertain PTAs that include IIAs. However, we choose to focus on BITs because of their relevance to ICSID and we have no reason to expect that accounting for multilateral agreements as well would substantively change our results re the affect of disputes on FDI or reputation. 
		}}
	\end{itemize}	
\end{enumerate}
