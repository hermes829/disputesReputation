one more that you appear to have overlooked: Aisbett/Busse/Nunnenkamp (2016)

\section{Reviewer 1}

\subsection{Major Comments}

\begin{enumerate}
	\item To what extent is the basic assumption plausible that ISDS claims tarnish reputation? A series of studies have shown that awareness of IIAs and more so of claims is very limited even among foreign investors - why would claims impact on reputation? Among well-informed investors, it has become known that a good number of claims are entirely without merits, so the fact that a state is hit by a claim does not necessarily mean that the state has done anything reproachable (the institution that has most suffered in terms of reputation lately is ISDS itself). There are so many more interactions between businesses and governments (and so few claims, overall and against individual countries) that contribute to reputation, that noise probably covers any sign that could come out of investment treaty claims. In all, the findings are likely to be artifacts. 
	\begin{itemize}
		\item \textcolor{blue}{ \emph{ hi world . }}
	\end{itemize}
	\item One key problem of the design is the use of FDI data (and in particular the highly volatile flow data) for the econometric analysis. Although often repeated, the use of this dataset is highly problematic for the purpose of the study, as has been recognized for a decade now (see first Robert E. LIPSEY (2007), ``Defining and measuring the location of FDI output'' Sjoerd BEUGELSDIJK/Jean-Francois HENNART/Arjen SLANGEN/Roger SMEETS (2010), ``Why and how FDI stocks are a biased measure of MNE affiliate activity''; and later Andrew KERNER (2014), ``What We Talk About When We Talk About Foreign Direct Investment'' and Andrew KERNER/Jane LAWRENCE (2012)). Many econometric studies that seek to assess treaty effects still use this data for lack of other available data or due to ignorance, but this is not a good reason to use this data without any discussion on its validity and implications for the exercise. 
	\begin{itemize}
		\item \textcolor{blue}{ \emph{ hi world . }}
	\end{itemize}
	\item Also: Why is the number of claims not normalised against the volume of investment that individual countries receive overall (a country that receives little investment from anywhere is unlikely to be exposed to a lot of claims, while a country that attracts a lot of foreign investment would normally be more likely to get hit by claims); those that perceive the reputation can be assumed to factor this in. Also, not all countries have concluded IIAs with countries from which they receive meaningful amounts of investment, so the exposure to such claims is very different. This is also a fact that would be known to those that know about claims. 
	\begin{itemize}
		\item \textcolor{blue}{ \emph{ hi world . }}
	\end{itemize}
	\item At FN46, the authors express the surprising view that ``we expect the number of ratified BITs to be positively related to reputation''. Many hold that the opposite is likely to be the case (at least among developing economies, but the text is unclear on whether advanced economies are included in the statement, given the statement after FN53). BITs would more likely be used by states to compensate for mixed reputation to international investors - see, e.g., the papers cited earlier at FN11 and FN12. 
	\begin{itemize}
		\item \textcolor{blue}{ \emph{ hi world . }}
	\end{itemize}
\end{enumerate}

\subsection*{Other Comments}

\begin{enumerate}
	\item When the distinctive features of ICSID are described, the second item does not set ICSID apart. In fact, all arbitration institutions and rules, in combination with the IIAs, provide binding and enforceable awards. The legal authority of ICSID, if such a thing exists, appears irrelevant, as decisions are taken by the same kind of arbitrators that also adjudicate disputes under other rules and institutions. ICSID itself only facilitates the adjudication process. 
	\begin{itemize}
		\item \textcolor{blue}{ \emph{ hi world . }}
	\end{itemize}
	\item Slicing off upper income nations (at FN40) for unspecific reasons (``significantly different role in the system'' - why?) is not a plausible and satisfying way to address this issue. The fact that advanced economies are increasingly defendants of treaty claims (without their reputation being tarnished) is interesting and questions the basic assumptions. Canada is a case in point: It got hit by a large number of claims, of which it lost some, and still does not have a ``bad'' reputation for foreign investors. How would you explain this fact?	
	\begin{itemize}
		\item \textcolor{blue}{ \emph{ hi world . }}
	\end{itemize}
	\item Figure 1 arguably has a normalisation problem: the overall number of newspaper articles referenced on LexisNexis probably goes up every year, so a potentially useful measure of public attention would be the priority - relative frequency of mentioning - that ``ICSID'' gets in the news. This could be measured as a percentage of articles mentioning ``ICSID'' in a given year in all referenced articles in that given year. It is likely to show quite a different graph, and is more meaningful that the absolute number. The axis-title ``frequency'' should also be amended, as a frequency cannot be expressed in absolute numbers (``occurrences'' would probably be better for this graph, but ``proportion'' or ``share of'' would be probably required if the graph was normalised as proposed here. 	
	\begin{itemize}
		\item \textcolor{blue}{ \emph{ hi world . }}
	\end{itemize}
	\item Why is population size (FN48) assumed to be positively correlated with international reputation? The countries with the largest populations do not normally feature on top in this regard. In the top 20, only three countries would be considered by most as having an above average reputation in foreign investors' eyes, and these are all advanced economies (and excluded from the study). 
	\begin{itemize}
		\item \textcolor{blue}{ \emph{ hi world . }}
	\end{itemize}
	\item A large number of claims are now brought under multilateral arrangements (e.g. Energy Charter Treaty, NAFTA, CAFTA-DR) that have almost identical features as BITs and FTAs with respect to investment protection. Focusing exclusively on bilateral arrangements is an interesting choice in this regard (but admittedly, UNCTAD does not provide the required dataset off the shelf). 		
	\begin{itemize}
		\item \textcolor{blue}{ \emph{ hi world . }}
	\end{itemize}	
\end{enumerate}
