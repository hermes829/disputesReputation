\section{Reviewer 2}

\subsection{Major Comments}

\begin{enumerate}
	\item Substantively, I invite the authors to revise and update the discussions in pages 5-7. The authors said that the previous studies have found that the simple fact that a private firm brings a claim against a state on potential treaty violations regardless of the actual verdicts damages the state’s reputation as a law-abiding member of the international community in the previous section. Now, the authors say that the ISDS process’s unique characteristics of being case-specific, decentralized, uncertain, and non-transparent may not lead to a significant reputation loss because reputations are sticky and constructed around multiple observations. These two claims seem not consistent. If only the fact that claims against a state are made matters, why do we need to care about the variations in specific designs in ISDS across treaties? If you are trying to reveal the inconsistency among the existing literature’s claims, please revise your writing in this part. Currently, it is not very clear what you are trying to establish in this part.
	\begin{itemize}
		\item \textcolor{blue}{ \emph{ 
		We thank the reviewer for these comments and have updated the relevant sections to better clearly state our arguments. 
		}}
	\end{itemize}
	\item For the sake of clarity, please state hypotheses in a separate section right below the theory part, for example. With the current format, it is somewhat hart to follow.	
	\begin{itemize}
		\item \textcolor{blue}{ \emph{ hi world . }}
	\end{itemize}
	\item If simply facing a dispute either at the ICSID nor at the Non-ICSID is not associated with a meaningful change in FDI flows, why does it matter in affecting the ICRG ratings? Could you elaborate on this gap more?	
	\begin{itemize}
		\item \textcolor{blue}{ \emph{
		In general, we think this is because a lot more goes into shaping FDI flows than reputation. 
		}}
	\end{itemize}
	\item Regarding Tables 2, 3, and 4, authors report the results with country fixed effects. However, they are only showing within country variations. Do you have the pooled results? If so and if they are similar to country fixed effects, please mention in briefly at least.
	\begin{itemize}
		\item \textcolor{blue}{ \emph{
		
		}}
	\end{itemize}
	\item What if you include cumulative ICSID (t-1) in the Models 1,2,(3,4) in Tables 2, 3, and 4? Recent counts of claims matter but you still need to control for the past history of claims in the model. And what about the potential endogeneity? It is possible that states with lower ICRG ratings tend to have more number of investment disputes.		
	\begin{itemize}
		\item \textcolor{blue}{ \emph{ hi world . }}
	\end{itemize}
	\item Explain the scale of the ICRG ratings in more detail to help the reader’s understanding of the results.	
	\begin{itemize}
		\item \textcolor{blue}{ \emph{ hi world . }}
	\end{itemize}
	\item What happens if you include ICSID and Non-ICSID in the same model so that you test their effects simultaneously?		
	\begin{itemize}
		\item \textcolor{blue}{ \emph{ hi world . }}
	\end{itemize}
	\item Why one point estimate is in blue? If this is an error, please correct it in Figure 4.	
	\begin{itemize}
		\item \textcolor{blue}{ \emph{ hi world . }}
	\end{itemize}
	\item Please rewrite the Introduction to clearly present what motivates your research, in what aspects you challenge the previous studies both theoretically and empirically, and what your arguments are and how you are going to prove them. The current format is not a very effective introduction for the readers.
	\begin{itemize}
		\item \textcolor{blue}{ \emph{
		Per reviewer two's request, we have thoroughly rewritten the introduction to better highlight the motivation of our work here and our the contribution of our work.
		}}
	\end{itemize}	
\end{enumerate}